%!TeX Options=--shell-escape
\documentclass{article}

\usepackage{amsmath, amssymb, hyperref, xcolor, minted, enumitem}

\hypersetup{%
    pdfstartview=FitH,
    bookmarksnumbered=true,
    bookmarksopen=true, 
    pdfborder=001,
    allcolors=blue,
    breaklinks=true
}%

\title{An Introduction to NormalOpt Collection \\ Version 1.0.0}
\author{Tony Xiang}

\begin{document}
\maketitle
\tableofcontents

\newpage

\section{Overall Introduction}
This collection is written inspired by Chen Baolin's \textit{Optimization Theory and Algorithms} and the optimization course in Wuhan University. As those algorithms are really fanscinating, we decide to implement them by myself. Before you use these MATLAB codes, you should be aware that MATLAB has its own Optimization Toolbox and it's very useful in most circumstances. Our work is not based on the toolbox, but based on MATLAB itself and another toolbox named Symbolic Math Toolbox to get the symbolic derivatives of functions as an input.

This is a collection of many useful algorithms. The names of each file show exactly what they mean. Files started with lowerbound letters are utilities which don't need be executed individually at most time. Files started with upperbound/capital letters are implemented algorithms which should be executed individually for use.

The whole collection is licensed under GNU General Public License Version 2.0. You can choose this license or any later version if you want to use our codes in your project. The full license will be posted at the end of this introduction.

\section{Utilities}
\textcolor{blue}{Keep in mind: Utilities don't need to be executed individually at most time.}

\subsection{Swap Function}
\begin{minted}[breaklines, breakanywhere]{matlab}
function [c, d] = swap(a, b)
\end{minted}

Swap two numbers. If $a, b$ are inputs, $b, a$ will be outputs. We use additional $c, d$ for this substitution, so the original $a, b$ is \textcolor{blue}{unchanged}.

\subsection{Derivative Functions}
\subsubsection{Get Derivative Function}
\begin{minted}[breaklines, breakanywhere]{matlab}
function tanFunct = getTangent(funct, dimension)
\end{minted}

\paragraph{Introduction}
Get the derivative function of a given function. Derivative is calculated per variable(partially). \textcolor{blue}{funct} is a \textbf{function handle} in MATLAB. \textcolor{blue}{funct} should be created as a vector, like this:
\begin{minted}[breaklines, breakanywhere]{matlab}
funct = @(x) x(1).^2 + 1./x(2) + x(3)   
\end{minted}

And \textcolor{blue}{dimension} is the variables in your function. For example, the function created above has $3$ dimensions. The output will be a $1 \times n$ cell which n is according to \textcolor{blue}{dimension}. The output shows the devirative of each variable using anonymous functions in the cell.

\paragraph{Algorithm}
The algorithm is quite simple here. We use Symbolic Math Toolbox in MATLAB to help out. 
\begin{enumerate}
    \item Convert input function to a symbolic function.
    \item Change variable style for convenience.
    \item Get derivatives by the toolbox.
    \item Convert the result to anonymous function, and put it in the cell.
\end{enumerate}

\paragraph{\textcolor{red}{Warnings}}
\begin{enumerate}
    \item \textcolor{blue}{funct} should have devirative at least.
    \item \textcolor{blue}{funct} and \textcolor{blue}{dimension} must match.
    \item \textcolor{blue}{dimension} will be $1$ if not assigned.
\end{enumerate}

\subsubsection{Get Second Order Derivative Function}
\begin{minted}[breaklines, breakanywhere]{matlab}
function tanFunct = getDoubleTangent(funct, dimension)
\end{minted}

\paragraph{Introduction}
This function is based on Get Derivative Function. The inputs are the same as what's mentioned in Get Derivative Function. The difference is that the function calculates second order derivatives of a given function, and the output is a $n \times n$ cell, which $n$ denotes to the \textcolor{blue}{dimension}. The formats are the same as Get Derivative Function.

\subsubsection{Get Derivative Value Function}
\begin{minted}[breaklines, breakanywhere]{matlab}
function tval = getTangentValue(tanFunct, point)
\end{minted}

\paragraph{Introduction}
This function should \textbf{receive devirative outputs} from Get Derivative Function or Get Second Order Derivative Function. It will calculate the value based on the \textcolor{blue}{point} assigned to the function and return a \textbf{matrix} output which is the same size as input. By executing it, the format will be convenient to use.

\subsection{Search Functions}
In this part, we are going to talk about one dimensional search on functions. The general aim is to solve:
\begin{equation*}
    \mbox{minimize } f(x), x \in \mathbb{R}^1 \mbox{ or bounded.}
\end{equation*}

\subsubsection{Golden Mean Method}
\begin{minted}[breaklines, breakanywhere]{matlab}
function point = searchGoldenMean(funct, start, stop, epsilon)
\end{minted}

\paragraph{Introduction}
Find the point which reaches the minimum value in a function. \textcolor{blue}{funct} is a \textbf{function handle} in MATLAB, an anonymous function will be the best choice. A single function script is OK. \textcolor{blue}{start and stop} is the search interval you want to use. \textcolor{blue}{epsilon} is the limit to stop the iteration.

\paragraph{Algorithm} 
The algorithm is implemented below, \textcolor{blue}{funct} is denoted as $f$, \textcolor{blue}{start} is denoted as $b$, \textcolor{blue}{end} is denoted as $e$, \textcolor{blue}{epsilon} is denoted as $\varepsilon$:
\begin{enumerate}
    \item set the GoldenMean value: $g = (\sqrt{5} - 1) / 2$. It's stored as a double.
    \item set $\lambda = b + (1 - g)(e - b)$, and $\lambda = b + g(e - b)$.
    \item If $e - b \geqslant \varepsilon$, go to Step 4, else go to Step 7.
    \item If $f(\lambda) > f(\mu)$, go to Step 5, else go to Step 6.
    \item Let $b = \lambda, \lambda = \mu, \mu = b + g(e - b)$, and go to Step 4.
    \item Let $e = \mu, \mu = \lambda, \lambda = b + (1 - g)(e - b)$, and go to Step 4.
    \item Return $(b + g) / 2$ as output.
\end{enumerate}

\paragraph{\textcolor{red}{Warnings}}
\begin{enumerate}
    \item \textcolor{blue}{epsilon} must be greater than $0$. 
    \item \textcolor{blue}{end} must be greater than \textcolor{blue}{start}.
    \item \textcolor{blue}{funct} must have only one minimum. The program will misbehave when there are multiple minimums.
\end{enumerate}

\paragraph{An example (not needed mostly)}
Suppose we want to get the minimum of $f(x) = x^4 - 6x^2 + 2x + 1, x in [0.5, 3]$ with error $\varepsilon = 10^{-6}$. We can use the function like this:
\begin{minted}[breaklines, breakanywhere]{matlab}
point = searchGoldenMean(@(x) x.^4 - 6*x.^2 + 2*x + 1, 0.5, 3, 1e-6)
\end{minted}

And the result is $1.6418$ which is the right answer.

\subsubsection{Tangent Method}
\begin{minted}[breaklines, breakanywhere]{matlab}
function point = searchTangent(funct, start, stop, epsilon)
\end{minted}

\paragraph{Introduction}
Find the point which reaches the minimum value in a function. \textcolor{blue}{funct, epsilon} is the same as what's mentioned in Golden Mean Method. \textcolor{blue}{start, stop} is your initial guess interval which the optimal point might be into. This function is based on Get Derivative Function.

\paragraph{Algorithm}
The algorithm is implemented below, the denotions are the same as what's mentioned in Golden Mean Method:
\begin{enumerate}
    \item Get the derivative of \textcolor{blue}{funct} at $e$, let it be $f'(e)$.
    \item If $|f'(e)| \geqslant \varepsilon$, go to Step 3, else go to Step 4.
    \item Let $t$ is a temporary variable. Let $t = e, e = e - \frac{e - b}{f'(e) - f'(b)} f'(e), b = t$, go to Step 2.
    \item Return $e$ as output.
\end{enumerate}

\paragraph{\textcolor{red}{Warnings}}
\begin{enumerate}
    \item \textcolor{blue}{epsilon} must be greater than $0$. 
    \item \textcolor{blue}{end} must be greater than \textcolor{blue}{start}.
    \item If \textcolor{blue}{funct} have more than one minimum, this function might not find the optimal value.
    \item The output might be outside the given \textcolor{blue}{start} and \textcolor{blue}{stop} interval. Narrower intervals will reach higher precision.
\end{enumerate}

\paragraph{An example (not needed mostly)}
Suppose we want to get the minimum of $f(x) = x^4 - 6x^2 + 2x + 1$, initial interval is $[0.5, 3]$ with error $\varepsilon = 10^{-6}$. We can use the function like this:
\begin{minted}[breaklines, breakanywhere]{matlab}
point = searchTangent(@(x) x.^4 - 6*x.^2 + 2*x + 1, 0.5, 3, 1e-6)
\end{minted}

And the result is $0.1683$ which isn't the right answer.

But if we narrow the interval to $[2, 3]$, the result comes back to $1.6418$.

\subsubsection{Newton Method}
\begin{minted}[breaklines, breakanywhere]{matlab}
function point = searchNewton(funct, initial, epsilon)
\end{minted}

\paragraph{Introduction}
Find the point which reaches the minimum value in a function. \textcolor{blue}{funct, epsilon} is the same as what's mentioned in Golden Mean Method. \textcolor{blue}{initial} is the initial value in the iteration. This function is based on Get Double Function Method.

\paragraph{Algorithm}
The algorithm is implemented below, \textcolor{blue}{funct} is denoted as $f$, \textcolor{blue}{initial} is denoted as $x$, \textcolor{blue}{epsilon} is denoted as $\varepsilon$:
\begin{enumerate}
    \item Get the derivative and the second derivative of \textcolor{blue}{funct} at $x$, let it be $f'(x)$ and $f''(x)$.
    \item If $|f'(x)| \geqslant \varepsilon$, go to Step 3, else go to Step 4.
    \item Update initial $x = x - \frac{f'(x)}{f''(x)}$, go to Step 2.
    \item Return $x$ as output.
\end{enumerate}

\paragraph{\textcolor{red}{Warnings}}
\begin{enumerate}
    \item \textcolor{blue}{epsilon} must be greater than $0$. 
    \item Warnings will be given when the second order of \textcolor{blue}{funct} is less than \textcolor{blue}{epsilon}. This is an ill conditioned function.
    \item If \textcolor{blue}{funct} have more than one minimum, this function might not find the optimal value.
    \item This method is \textbf{very} sensitive to \textcolor{blue}{initial} point. And \textcolor{blue}{funct} need to be $C^2$.
\end{enumerate}

\paragraph{An example (not needed mostly)}
Suppose we want to get the minimum of $f(x) = x^4 - 6x^2 + 2x + 1$, initial point is $0.5$ with error $\varepsilon = 10^{-6}$. We can use the function like this:
\begin{minted}[breaklines, breakanywhere]{matlab}
point = searchNewton(@(x) x.^4 - 6*x.^2 + 2*x + 1, 0.5, 1e-6)
\end{minted}

And the result is $0.1683$ which isn't the right answer.

But if we set the initial point to $2$, the result comes back to $1.6418$.

\subsubsection{Parabola Method}
\begin{minted}[breaklines, breakanywhere]{matlab}
function point = searchParabola(funct, start, mid, stop, epsilon)
\end{minted}

\paragraph{Introduction}
Find the point which reaches the minimum value in a function. \textcolor{blue}{funct, epsilon} is the same as what's mentioned in Golden Mean Method. \textcolor{blue}{start, mid, stop} is the initial guess to start the iteration. This function uses swap function.

\paragraph{Algorithm}
The algorithm is implemented below, \textcolor{blue}{funct} is denoted as $f$, \textcolor{blue}{start} is denoted as $b$, \textcolor{blue}{mid} is denoted as $m$, \textcolor{blue}{stop} is denoted as $e$, \textcolor{blue}{epsilon} is denoted as $\varepsilon$:
\begin{enumerate}
    \item Calculate values at $b, m, e$, and let them $f(b), f(m), f(e)$. Let $v = \min\{f(b)$, $f(m)$, $f(e)\}$.
    \item If $v = f(b)$, swap $b, m$, else if $v = f(e)$, swap $e, m$. Set a decrease variable $d = +\infty$.
    \item If $d \geqslant 0$, go to Step 4, else go to Step 6.
    \item Recalculate $f(b), f(m), f(e)$, and construct a parabola $P$ by $(b, f(b))$, $(m, f(m))$, $(e, f(e))$. Get the minimum of $P$ at $(x^*, f(x^*))$. Update decrease variable $d = \min{\{f(b), f(m), f(e)\}} - f(x^*)$.
    \item Add point $x^*$ into $\{b, m, e\}$, and remove the point which reaches highest value. Plus, raise an error if $m$ is removed. Go to Step 3.
    \item Return $x^*$ as output.
\end{enumerate}

\paragraph{\textcolor{red}{Warnings}}
\begin{enumerate}
    \item \textcolor{blue}{epsilon} must be greater than $0$. 
    \item The \textcolor{blue}{funct} $f$ and \textcolor{blue}{start, mid, stop} $b, m, e$ \textbf{must satisfy} $f(b) > f(m)$ and $f(e) > f(m)$.
    \item If \textcolor{blue}{funct} have more than one minimum, this function might not find the optimal value.
    \item Narrower intervals will reach higher precision.
\end{enumerate}

\paragraph{An example (not needed mostly)}
Suppose we want to get the minimum of $f(x) = x^4 - 6x^2 + 2x + 1$, initial guess is $0.5, 2, 3$ with error $\varepsilon = 10^{-6}$. We can use the function like this:
\begin{minted}[breaklines, breakanywhere]{matlab}
point = searchParabola(@(x) x.^4 - 6*x.^2 + 2*x + 1, 0.5, 2, 3, 1e-6)
\end{minted}

And the result is $1.6415$ which is near the right answer.

\subsubsection{Cubic Interpolation Method}
\begin{minted}[breaklines, breakanywhere]{matlab}
function point = searchInterp(funct, start, stop, epsilon)
\end{minted}

\paragraph{Introduction}
Find the point which reaches the minimum value in a function. \textcolor{blue}{funct, epsilon} is the same as what's mentioned in Golden Mean Method. \textcolor{blue}{start, stop} is the initial interval to start the iteration. This function is based on the Get Derivative Function.

\paragraph{Algorithm}
The algorithm is implemented below, \textcolor{blue}{funct} is denoted as $f$, the derivative of $f$ is denoted as $f'$, \textcolor{blue}{start} is denoted as $b$, \textcolor{blue}{stop} is denoted as $e$, \textcolor{blue}{epsilon} is denoted as $\varepsilon$:
\begin{enumerate}
    \item If $|e - b| \geqslant 0$, go to Step 2, else go to Step 4.
    \item Let $s = 3 \frac{f(e) - f(b)}{e - b}$, $z = s - f'(e) - f'(b)$, $w = \sqrt{z^2 - f'(b) f'(e)}$. Let $x^* = b + (e - b)(1 - \frac{f'(e) + w + z}{f'(e) - f'(b) + 2w})$.
    \item If $f'(x^*) < 0$, $b = x^*$, else if $f'(x^*) > 0$, $e = x^*$, else go to Step 4.
    \item Return $x^*$ as output.
\end{enumerate}

\paragraph{\textcolor{red}{Warnings}}
\begin{enumerate}
    \item \textcolor{blue}{epsilon} must be greater than $0$. 
    \item \textcolor{blue}{end} must be greater than \textcolor{blue}{start}.
    \item The \textcolor{blue}{funct} $f$ and initial guess $b, e$ must satisfy $f'(b) < 0$ and $f'(e) > 0$.
    \item If \textcolor{blue}{funct} have more than one minimum, this function might not find the optimal value.
\end{enumerate}

\paragraph{An example (not needed mostly)}
Suppose we want to get the minimum of $f(x) = x^4 - 6x^2 + 2x + 1, x \in [0.5, 3]$ with error $\varepsilon = 10^{-6}$. We can use the function like this:
\begin{minted}[breaklines, breakanywhere]{matlab}
point = searchInterp(@(x) x.^4 - 6*x.^2 + 2*x + 1, 0.5, 3, 1e-6)
\end{minted}

And the result is $1.6418$ which is the right answer. Actually this method is very stable.

\subsubsection{Search a Valid Interval}
\begin{minted}[breaklines, breakanywhere]{matlab}
function [start, stop] = searchValidInterval(funct, initial, step)
\end{minted}

\paragraph{Introduction}
Unlike the various searching methods above, this function is made to search for a valid interval to apply the methods on. \textcolor{blue}{funct} is the function you want to search for a valid interval. \textcolor{blue}{initial} is the initial guess to start the iteration. \textcolor{blue}{step} is the initial step for a search, and it's adaptive to the input \textcolor{blue}{funct}. This function has two outputs, \textcolor{blue}{start} and \textcolor{blue}{stop} which stores the interval for further use. This function uses swap function.

\paragraph{Algorithm}
The algorithm is implemented below, \textcolor{blue}{funct} is denoted as $f$, \textcolor{blue}{initial} is denoted as $x$, \textcolor{blue}{start} is denoted as $b$, \textcolor{blue}{stop} is denoted as $e$ (Apparently, $e = b + x$), \textcolor{blue}{step} is denoted as $l$:
\begin{enumerate}
    \item If $f(e) \geqslant f(b)$, $l = -l$, and swap $b, e$. And Set a temporary variable $m = s$.
    \item If $f(e) < f(b)$, go to Step 3, else go to Step 4.
    \item Let $b = m, m = e, l = 2l, e = m + l$, go to Step 2.
    \item If $b > e$, swap $b, e$. Return $b, e$ as output.
\end{enumerate}

\paragraph{\textcolor{red}{Warnings}}
\begin{enumerate}
    \item \textcolor{blue}{step} can't be $0$.
    \item The output of this function is based on $f, l$. Smaller $l$ will reach higher precision, but it's time-consuming.
    \item This search method is often applied on $\mathbb{R}^1$.
    \item This function will misbehave when \textcolor{blue}{funct} is monotone. To handle this, if $l > 10^{10}$, we decide to return $[-\infty, \infty]$, and throw an \textcolor{red}{error} message in search programs afterwards if no boundary conditions are set.
\end{enumerate}

\paragraph{Example}
Suppose we want to get a valid interval of $f(x) = x^4 - 6x^2 + 2x + 1, x \in \mathbb{R}^1$ with an initial point $x = 0$ and an initial step $l = 10^{-3}$. We can use the function like this:
\begin{minted}[breaklines, breakanywhere]{matlab}
[start, stop] = searchValidInterval(@(x) x.^4 - 6*x.^2 + 2*x + 1, 0, 1e-3)
\end{minted}

We get $[-4.0940, -1.0220]$ as a result. If you use the methods mentioned above, you will get $x = -1.8100$ which $f(x)$ will be smaller than the search on $[0.5, 3]$. So this function often searches broader area if the input \textcolor{blue}{funct} is good enough.

\subsubsection{Universal Search}
\begin{minted}[breaklines, breakanywhere]{matlab}
function point = UniversalSearch(funct, initial, epsilon, step, [method], [lb], [ub])
\end{minted}

\paragraph{Introduction}
This function works as a wrapper which can call all of the searching methods and find a valid interval automatically. \textcolor{blue}{funct, initial, epsilon, step} is the same as what's mentioned before. And we are going to take a look at three optional arguments: \textcolor{blue}{method, lb} and \textcolor{blue}{ub}. \textbf{Optional arguments will be wrapped with brankets for difference.}

\textcolor{blue}{method} is the searching method you want to use. Available methods are: 'goldenmean'(default method), 'tangent', 'newton', 'parabola' and 'interp'. They match Golden Mean Method, Tangent Method, Newton Method, Parabola Method and Cubic Interpolation Method in order. \textcolor{blue}{lb} is the lowerbound of $x$, and \textcolor{blue}{ub} is the upperbound of $x$. In a word, \textcolor{blue}{lb} $\leqslant x \leqslant$ \textcolor{blue}{ub}.

\paragraph{\textcolor{red}{Warnings}}
\begin{enumerate}
    \item \textcolor{blue}{lb} and \textcolor{blue}{ub} is \textbf{only} compatible with 'goldenmean' method.
    \item 'newton' and 'parabola' method might be incompatible with this wrapper as the principle inside the algorithms is unlike that in 'goldenmean', 'parabola' and 'interp'.
    \item 'parabola' method might be instable because three valid points are needed.
\end{enumerate}

This function can be used individually for some one dimensional functions, but it has more use in further sections. As examples are mentioned individually before, there are no examples here.

\subsection{Barrier Functions}
\textcolor{blue}{You don't need to execute functions in this section at all.}

\subsubsection{Get Outer Barrier Function}
\begin{minted}[breaklines, breakanywhere]{matlab}
function outerBarrierFunct = getOuterBarrier(constraint, equality, dimension)
\end{minted}
\paragraph{Introduction}
Find the outer barrier function in a cell of inequalities and equalities. \textcolor{blue}{constraint} is a cell of inequalities, \textcolor{blue}{equality} is a cell of equalities, \textcolor{blue}{dimension} is the dimension of \textcolor{blue}{initial} which doesn't appear here explicitly.

\paragraph{Algorithm}
The algorithm is implemented below, a detailed introduction is in the Constrained Optimization section, Outer Barrier Method part.
\begin{enumerate}
    \item Convert input function to a symbolic function.
    \item Change variable style for convenience.
    \item Get outer barrier functions by the toolbox.
    \item Convert the result to anonymous function, and put it in the cell.
\end{enumerate}

\subsubsection{Get Inner Barrier Function}
\begin{minted}[breaklines, breakanywhere]{matlab}
function innerBarrierFunct = getInnerBarrier(constraint, dimension, [type])
\end{minted}
\paragraph{Introduction}
Find the inner barrier function in a cell of inequalities. \textcolor{blue}{constraint} is a cell of inequalities, \textcolor{blue}{dimension} is the dimension of \textcolor{blue}{initial} which doesn't appear here explicitly, \textcolor{blue}{type} is the type to create an inner barrier function. Type $1$ is reciprocal style, and type $2$ is logarithm style.

\paragraph{Algorithm}
The algorithm is implemented below, a detailed introduction is in the Constrained Optimization section, Inner Barrier Method part.
\begin{enumerate}
    \item Convert input function to a symbolic function.
    \item Change variable style for convenience.
    \item Get inner barrier functions by the toolbox.
    \item Convert the result to anonymous function, and put it in the cell.
\end{enumerate}

\section{Linear Programming}
\subsection{Simplex Method}
In the following parts, we are going to talk about Simplex Method. We divide this method into two parts. Simplex Core is used to solve the following LP problem:
\begin{align*}
    \mbox{minimize } & f = \mathbf{c}\mathbf{x} \\
    \mbox{s.t. } & A\mathbf{x} = \mathbf{b} \\
                 & \mathbf{x} \geqslant 0 \\
                 & \mathbf{b} \geqslant 0
\end{align*}

And the general Simplex Solver is used to solve the following LP problem:
\begin{align*}
    \mbox{minimize } & f = \mathbf{c}\mathbf{x} \\
    \mbox{s.t. } & A\mathbf{x} \leqslant \mathbf{b} \\
                 & A_{\mbox{eq}} \mathbf{x} = \mathbf{b}_{\mbox{eq}} \\
                 & \mbox{lb} \leqslant \mathbf{x} \leqslant \mbox{ub}
\end{align*}

Simplex Core is a special condition of Simplex Solver.

\subsubsection{Simplex Core}
\begin{minted}[breaklines, breakanywhere]{matlab}
function [xval, fval] = SimplexMethod(c, A, b, [epsilon])    
\end{minted}
\paragraph{Introduction}
In the program, \textcolor{blue}{c} is $\mathbf{c}$, \textcolor{blue}{A} is $A$, and \textcolor{blue}{b} is $\mathbf{b}$ in the former LP problem. And \textcolor{blue}{epsilon} is used to control the limit to find a starting point.

\paragraph{Algorithm}
This algorithm is not easy to describe here. We are only giving a brief introduction.
\begin{enumerate}
    \item Find out a valid solution using a table.
    \item If no solutions are found, print an error message, else find out a optimal solution.
\end{enumerate}

\paragraph{\textcolor{red}{Warnings}}
\begin{enumerate}
    \item Please make sure all of your inputs have the right size/dimension.
    \item \textcolor{red}{[Bug Warning]} This program can't cover all of the LP problems.
    \item The table is manipulated by row transformation and pivoting in every iteration.
\end{enumerate}

\subsubsection{Simplex Solver}
\begin{minted}[breaklines, breakanywhere]{matlab}
function [xval, fval] = SimplexSolver(c, A, b, [Aeq], [beq], [lb], [ub], [epsilon])    
\end{minted}
In the program, \textcolor{blue}{c} is $\mathbf{c}$, \textcolor{blue}{A} is $A$, and \textcolor{blue}{b} is $\mathbf{b}$, \textcolor{blue}{Aeq} is $A_{\mbox{eq}}$, \textcolor{blue}{beq} is $\mathbf{b}_{\mbox{eq}}$, in the former LP problem (lb and ub is unchanged by the way). And \textcolor{blue}{epsilon} is used to control the limit to find a starting point.

\paragraph{Algorithm}
This algorithm is not easy to describe here. We are only giving a brief introduction.
\begin{enumerate}
    \item Convert the LP in Simplex Solver to Simplex Method by adding some additional variables. 
    \item Find out a valid solution using a table.
    \item If no solutions are found, print an error message, else find out a optimal solution.
    \item The table is manipulated by row transformation and pivoting in every iteration.    
\end{enumerate}

\paragraph{\textcolor{red}{Warnings}}
\begin{enumerate}
    \item Please make sure all of your inputs have the right size/dimension.
    \item \textcolor{red}{[Bug Warning]} This program can't cover all of the LP problems.
\end{enumerate}

\paragraph{Example}
Suppose we are going to solve the following LP:
\begin{align*}
    \mbox{minimize } &  -x_1 - 8x_2 - 5x_3 - 6x_4 \\
    \mbox{s.t. } & x_1 + 4x_2 + 5x_3 + 2x_4 \leqslant 7 \\
                 & 2x_1 + 3x_2 \leqslant 6 \\
                 & 5x_1 + x_2 \leqslant 5 \\
                 & 3x_3 + 4x_4 \geqslant 12 \\
                 & x_3 \leqslant 4 \\
                 & x_4 \leqslant 3 \\
                 & x_j \geqslant 0, \quad j = 1, \cdots 4
\end{align*}

We can use the function like this:
\begin{minted}[breaklines, breakanywhere]{matlab}
[xval, fval] = SimplexSolver([-1 -8 -5 -6], [1 4 5 2; 2 3 0 0; 5 1 0 0; 0 0 -3 -4], [7 6 5 -12]', [], [], [0 0 0 0]', [+inf, +inf, 4, 3]')
\end{minted}

And the function returns xval $=[0, 0.2500, 0.0000, 3.0000]^T$ and fval$=-20.0000$ as a result, which is the right answer.

\section{Unconstrained Optimization}
\textcolor{blue}{Keep in mind: It's important to focus on \textbf{step} and \textbf{direction} here.}

In this section, we are going to talk about higher dimensional optimization on unconstrained functions. The general aim is to solve:
\begin{equation*}
    \mbox{minimize } f(\mathbf{x}), x \in \mathbb{R}^n.
\end{equation*}

\subsection{Based on Derivatives}
\subsubsection{Fastest Descent Method}
\begin{minted}[breaklines, breakanywhere]{matlab}
function [xval, fval] = FastestDescent(funct, initial, [epsilon], [step])
\end{minted}

\paragraph{Introduction}
Find the vector which reaches the minimum value in a function. \textcolor{blue}{funct, initial, epsilon, step} is the same as what's mentioned in the Search Functions part. This function has two outputs: \textcolor{blue}{xval} stands for a vector to reach the minimum value, and \textcolor{blue}{fval} stands for the minimum value. This function uses Golden Mean Method to search for the search step, and uses Get Derivative Function to search for direction.

\paragraph{Algorithm}
The algorithm is implemented below, \textcolor{blue}{funct} is denoted as $f$, the gradient of $f$ is denoted as $\nabla f$, \textcolor{blue}{initial} is denoted as $\mathbf{x}$, \textcolor{blue}{step} is denoted as $l$, \textcolor{blue}{epsilon} is denoted as $\varepsilon$:
\begin{enumerate}
    \item If $||\nabla f(\mathbf{x})|| \geqslant \varepsilon$, go to Step 2, else go to Step 4.
    \item Let searching direction $\mathbf{d} = -\nabla f(\mathbf{x})$, and searching the point $\lambda$ which minimizes $f(\mathbf{x} + \lambda \mathbf{d})$.
    \item Update value $\mathbf{x} = \mathbf{x} + \lambda \mathbf{d}$ using $\lambda, \mathbf{d}$ from Step 2, and go to Step 1.
    \item Return $\mathbf{x}, f(\mathbf{x})$ as outputs.
\end{enumerate}

\paragraph{\textcolor{red}{Warnings}}
\begin{enumerate}
    \item \textcolor{blue}{step} must be greater than $0$. Its default value is $10^{-3}$.
    \item \textcolor{blue}{epsilon} must be greater than $0$. Its default value is $10^{-6}$.
    \item Some functions are unable to be optimized. If you meet an error says 'Infinite boundary found'. It's likely that the function doesn't have a minimum (which means it minimum is $-\infty$).
    \item Try to make $\varepsilon$ larger if it takes a long time to calculate (or run into a dead loop). If search step $\lambda$ reaches $0$, we will forcibly terminate the iteration.
\end{enumerate}

\paragraph{Example}
Suppose we want to get the minimum of $f(x_1, x_2) = 2 x_1^2 + 3 x_2^2 - 4 x_1 x_2 + 6 x_1 - 9 x_2 + 2$ using default settings with an initial point $[-1, 0]$. We can use the function like this:
\begin{minted}[breaklines, breakanywhere]{matlab}
[xval, fval] = FastestDescent(@(x) 2 * x(1)^2 + 3 * x(2)^2 -4 * x(1) * x(2) + 6 * x(1) - 9 * x(2) + 2, [-1, 0])
\end{minted}

And the function returns $\mathbf{x} = [-0.0000, 1.5000], f(\mathbf{x}) = -4.7500$ as a result which is the right answer.

\subsubsection{Conjugate Gradient Method}
\begin{minted}[breaklines, breakanywhere]{matlab}
function [xval, fval] = ConjugateGradient(funct, initial, [epsilon], [step])
\end{minted}

\paragraph{Introduction}
Find the vector which reaches the minimum value in a function. The inputs and outputs are consistent with that in Fastest Descent. The direction is based on conjugate vectors.

\paragraph{Algorithm}
The algorithm is implemented below, \textcolor{blue}{funct} is denoted as $f$, the gradient of $f$ is denoted as $\nabla f$, \textcolor{blue}{initial} is denoted as $\mathbf{x}$, \textcolor{blue}{step} is denoted as $l$, \textcolor{blue}{epsilon} is denoted as $\varepsilon$:
\begin{enumerate}
    \item If $||\nabla f(\mathbf{x})|| \geqslant \varepsilon$, go to Step 2, else go to Step 6.
    \item Let an addition variable $\mathbf{y} = \mathbf{x}$ and calculate the gradient $\mathbf{d} = -\nabla f(\mathbf{y})$. Add a counting variable $c = 0$.
    \item If $||\nabla f(\mathbf{y})|| \geqslant \varepsilon$ and $c$ is smaller than the dimension of the function, go to Step 4, else go to Step 1.
    \item Search the point $\lambda$ which minimizes $f(\mathbf{y} + \lambda \mathbf{d})$.
    \item Update value $\mathbf{y'} = \mathbf{y} + \lambda \mathbf{d}$, $\beta = \frac{||\nabla f(\mathbf{y'})||^2}{||\nabla f(\mathbf{y})||^2}$, $\mathbf{d} = -\nabla f(\mathbf{y}) + \beta \mathbf{d}$. Break this iteration if $ \mathbf{d}\nabla f(\mathbf{d}) \geqslant 0$, else $c = c + 1, \mathbf{y} = \mathbf{y'}$ and go to Step 3.
    \item Return $\mathbf{x}, f(\mathbf{x})$ as outputs.
\end{enumerate}

\paragraph{\textcolor{red}{Warnings}}
\begin{enumerate}
    \item \textcolor{blue}{step} must be greater than $0$. Its default value is $10^{-3}$.
    \item \textcolor{blue}{epsilon} must be greater than $0$. Its default value is $10^{-6}$.
    \item Some functions are unable to be optimized. If you meet an error says 'Infinite boundary found'. It's likely that the function doesn't have a minimum (which means it minimum is $-\infty$).
    \item Try to make $\varepsilon$ larger if it takes a long time to calculate (or run into a dead loop). If search step $\lambda$ reaches $0$, we will forcibly terminate the iteration.
\end{enumerate}

\paragraph{Example}
Suppose we want to get the minimum of $f(x_1, x_2) = 2 x_1^2 + 3 x_2^2 - 4 x_1 x_2 + 6 x_1 - 9 x_2 + 2$ using default settings with an initial point $[-1, 0]$. We can use the function like this:
\begin{minted}[breaklines, breakanywhere]{matlab}
[xval, fval] = ConjugateGradient(@(x) 2 * x(1)^2 + 3 * x(2)^2 -4 * x(1) * x(2) + 6 * x(1) - 9 * x(2) + 2, [-1, 0])
\end{minted}

And the function returns $\mathbf{x} = [-0.0000, 1.5000], f(\mathbf{x}) = -4.7500$ as a result which is the right answer.

\subsubsection{Newton Method}
\begin{minted}[breaklines, breakanywhere]{matlab}
function [xval, fval] = NewtonMethod(funct, initial, [epsilon])
\end{minted}

\paragraph{Introduction}
Find the vector which reaches the minimum value in a function. \textcolor{blue}{funct, initial, epsilon} is the same as what's mentioned in the Search Function part. You don't need an initial step in this function. Newton method is an extension of the former search method.

\paragraph{Algorithm}
The algorithm is implemented below, \textcolor{blue}{funct} is denoted as $f$, the gradient of $f$ is denoted as $\nabla f$, the second order gradient of $f$ is denoted as $\nabla^2 f$, \textcolor{blue}{initial} is denoted as $\mathbf{x}$, \textcolor{blue}{epsilon} is denoted as $\varepsilon$:
\begin{enumerate}
    \item If $||\nabla f(\mathbf{x})|| \geqslant \varepsilon$, go to Step 2, else go to Step 3.
    \item $\mathbf{x} = \mathbf{x} - \nabla f(\mathbf{x}) (\nabla^2 f(\mathbf{x}))^{-1}$, go to Step 1.
    \item Return $\mathbf{x}, f(\mathbf{x})$ as outputs.
\end{enumerate}

\paragraph{\textcolor{red}{Warnings}}
\begin{enumerate}
    \item This method is instable as the Hessian Matrix $\nabla^2 f(x)$ is very likely to be singular.
    \item \textcolor{blue}{epsilon} must be greater than $0$. Its default value is $10^{-6}$.
\end{enumerate}

\paragraph{Example}
Suppose we want to get the minimum of $f(x_1, x_2) = 2 x_1^2 + 3 x_2^2 - 4 x_1 x_2 + 6 x_1 - 9 x_2 + 2$ using default settings with an initial point $[-1, 0]$. We can use the function like this:
\begin{minted}[breaklines, breakanywhere]{matlab}
[xval, fval] = NewtonMethod(@(x) 2 * x(1)^2 + 3 * x(2)^2 -4 * x(1) * x(2) + 6 * x(1) - 9 * x(2) + 2, [-1, 0])
\end{minted}

And the function returns $\mathbf{x} = [0.0000, 1.5000], f(\mathbf{x}) = -4.7500$ as a result which is the right answer.

\subsubsection{DFP and BFGS Method}
\begin{minted}[breaklines, breakanywhere]{matlab}
function [xval, fval] = DFPMethod(funct, initial, [epsilon], [step], [initialMatrix])
function [xval, fval] = BFGSMethod(funct, initial, [epsilon], [step], [initialMatrix])
\end{minted}

\paragraph{Introduction}
Find the vector which reaches the minimum value in a function. The inputs and outputs are consistent with that in Fastest Descent, and there is another optional input variable named \textcolor{blue}{initialMatrix}. As the Hessian Matrix is very likely to be singular in Newton Method, DFP Method and BFGS Method is two simliar approximation of Hessian Matrix. By these methods, the functions are much more stable than Newton Method.

\paragraph{Algorithm}
The algorithm is implemented below, \textcolor{blue}{funct} is denoted as $f$, the gradient of $f$ is denoted as $\nabla f$, \textcolor{blue}{initial} is denoted as $\mathbf{x}$, \textcolor{blue}{step} is denoted as $l$, \textcolor{blue}{epsilon} is denoted as $\varepsilon$, \textcolor{blue}{initialMatrix} is denoted as $A$. First take a look at DFP method:

\begin{enumerate}
    \item If $||\nabla f(\mathbf{x})|| \geqslant \varepsilon$, go to Step 2, else go to Step 6.
    \item Let direction $\mathbf{d} = -\nabla f(\mathbf{x}) A$. And search the point $\lambda$ which minimizes $f(\mathbf{x} + \lambda \mathbf{d})$. Set a counter $c$.
    \item If $c$ is smaller than the dimension of the function, go to Step 4, else go to Step 5.
    \item Let $\mathbf{x'} = \mathbf{x} + \lambda \mathbf{d}, \mathbf{p} = \mathbf{x'} - \mathbf{x}, \mathbf{q} = \nabla f(\mathbf{x'}) - \nabla f(\mathbf{x})$. 

    For DFP Method, update $A$ using the following formula:
    \begin{equation*}
        A = A + \frac{p^T p}{p q^T} - \frac{A q^T q A}{q A q^T}
    \end{equation*}

    For BFGS Method, update $A$ using the following formula:
    \begin{equation*}
        A = A + \frac{1 + q A q^T}{p q^T} \frac{p^T p}{p q^T} - \frac{p^T q A + A q^T p}{p q^T}
    \end{equation*}

    And then, let $c = c + 1$, and go to Step 1.
    \item Restore $A$ to its original value, and reset the counter $c = 0$, go to Step 1.
    \item Return $\mathbf{x}, f(\mathbf{x})$ as outputs.
\end{enumerate}

\paragraph{\textcolor{red}{Warnings}}
\begin{enumerate}
    \item \textcolor{blue}{step} must be greater than $0$. Its default value is $10^{-3}$.
    \item \textcolor{blue}{epsilon} must be greater than $0$. Its default value is $10^{-6}$.
    \item \textcolor{blue}{initialMatrix} has a default value, $A = I$, and it must be \textbf{symmetric}.
    \item Some functions are unable to be optimized. If you meet an error says 'Infinite boundary found'. It's likely that the function doesn't have a minimum (which means it minimum is $-\infty$).
    \item Try to make $\varepsilon$ larger if it takes a long time to calculate (or run into a dead loop). If search step $\lambda$ reaches $0$, we will forcibly terminate the iteration.
\end{enumerate}

\paragraph{Example}
Suppose we want to get the minimum of $f(x_1, x_2) = 2 x_1^2 + 3 x_2^2 - 4 x_1 x_2 + 6 x_1 - 9 x_2 + 2$ using default settings with an initial point $[-1, 0]$ and an initial matrix $A = diag(1, 2)$. We can use the function like this:
\begin{minted}[breaklines, breakanywhere]{matlab}
[xval, fval] = DFPMethod(@(x) 2 * x(1)^2 + 3 * x(2)^2 -4 * x(1) * x(2) + 6 * x(1) - 9 * x(2) + 2, [-1, 0], 1e-6, 1e-3, [1 0; 0 2])
[xval, fval] = BFGSMethod(@(x) 2 * x(1)^2 + 3 * x(2)^2 -4 * x(1) * x(2) + 6 * x(1) - 9 * x(2) + 2, [-1, 0], 1e-6, 1e-3, [1 0; 0 2])
\end{minted}

And the function returns $\mathbf{x} = [-0.0000, 1.5000], f(\mathbf{x}) = -4.7500$ as a result which is the right answer.

\subsubsection{Further Reading}
\paragraph{Quadratic Terminability} 
This means when an algorithm is applied on a \textbf{quadratic convex} function, a local optimal solution can be reached in limited iterations. Conjugate Gradient Method, Newton Method, DFP Method and BFGS Method has this property.

\paragraph{Broyden Group}
DFP Method and BFGS Method is very simliar, so there must be a way to combine them. We call this combination Broyden Group. Every function in Broyden Group has the following update method:
\begin{equation*}
    A = \varphi A_{BFGS} + (1 - \varphi) A_{DFP}
\end{equation*}

And every function in it has quadratic terminability. Our codes can be extended to use Broyden Group:
\begin{enumerate}
    \item Target these lines on DFP and BFGS method:
    \begin{minted}[breaklines, breakanywhere]{matlab}
% DFP Method
Hval = Hval + pval' * pval / (pval * qval') - Hval * (qval') * qval * Hval / (qval * Hval * qval');
% BFGS Method
Hval = Hval + (1 + qval * Hval * qval' / (pval * qval')) * (pval' * pval) / (pval * qval') - (pval' * qval * Hval + Hval * qval' * pval) / (pval * qval');
    \end{minted}
    \item Use a unifrom random number in $[0, 1]$ to represent $\varphi$.
    \item Combine the lines to one single line, together with $\varphi$.
\end{enumerate}

We are not going to implement this algorithm, so do it yourself if you want to have a try.

\paragraph{Changing Search Method}
These function are very limited in some conditions, and Golden Mean Method is forcibly applied to all of the functions in this section. If you really want to change the searching method, try to change some codes like this:
\begin{enumerate}
    \item Target at this line:
    \begin{minted}[breaklines, breakanywhere]{matlab}
lambda = UniversalSearch(dec_funct, 0, epsilon, step, 'goldenmean', 0, +inf);
    \end{minted}
    \item Change the line to:
    \begin{minted}[breaklines, breakanywhere]{matlab}
lambda = UniversalSearch(dec_funct, 0, epsilon, step, '...');
    \end{minted}
    \item Fill in the $\dots$ with the method you want. 
\end{enumerate}

Please note that these functions might misbehave after changes are made.

\subsection{Not Based on Derivatives}
\subsubsection{Hooke Jeeves Method}
\begin{minted}[breaklines, breakanywhere]{matlab}
function [xval, fval] = HookeJeeves(funct, initial, [epsilon], [dirMatrix], [delta], [alpha], [beta])
\end{minted}

\paragraph{Introduction}
Find the vector which reaches the minimum value in a function. This function doesn't need derivative of \textcolor{blue}{funct}. And \textcolor{blue}{funct, initial, epsilon} is the same as what's mentioned before. \textcolor{blue}{dirMatrix} is the initial searching direction matrix. \textcolor{blue}{delta} is the initial move step. \textcolor{blue}{alpha} is a multiply factor. \textcolor{blue}{beta} is a shrinking factor. As this method doesn't depend on derivative of a function, it might be slower than some methods based on derivatives.

\paragraph{Algorithm}
The algorithm is implemented below, \textcolor{blue}{funct} is denoted as $f$, \textcolor{blue}{initial} is denoted as $\mathbf{x}$, \textcolor{blue}{epsilon} is denoted as $\varepsilon$, \textcolor{blue}{dirMatrix} is denoted as $A$, \textcolor{blue}{delta} is denoted as $\delta$, \textcolor{blue}{alpha} is denoted as $\alpha$, \textcolor{blue}{beta} is denoted as $\beta$:
\begin{enumerate}
    \item Let a temporary variable $\mathbf{y} = \mathbf{x}$. If $\delta \geqslant \varepsilon$, go to Step 2, else go to Step 6.
    \item Check out the minimum in $f(\mathbf{y} + \delta A_i), f(\mathbf{y} - \delta A_i), \mathbf{y}$ in every direction ($A_i$ is the $i$th row of $A$), and moves $\mathbf{y}$ to the minimum one.
    \item If $f(\mathbf{y}) < f(\mathbf{x})$, go to Step 4, else go to Step 5.
    \item Let $\mathbf{x'} = \mathbf{y}, \mathbf{y} = \mathbf{x'} + \alpha(\mathbf{x'} - \mathbf{x}), \mathbf{x} = \mathbf{x'}$, and go to Step 1.
    \item Let $\delta = \beta \delta, \mathbf{y} = \mathbf{x}$, and go to Step 1.
    \item Return $\mathbf{x}, f(\mathbf{x})$ as outputs.
\end{enumerate}

\paragraph{\textcolor{red}{Warnings}}
\begin{enumerate}
    \item \textcolor{blue}{alpha} must be greater than $1$. Its default value is $2$.
    \item \textcolor{blue}{beta} must be between $0$ and $1$, excluding $0$ and $1$. Its default value is $0.5$.
    \item \textcolor{blue}{delta} must be greater than 0. Its default value is $1$.
    \item \textcolor{blue}{epsilon} must be greater than $0$. Its default value is $10^{-6}$.
    \item \textcolor{blue}{dirMatrix} must be a $n \times n$ matrix. Its default value is $A = I$.
\end{enumerate}

\paragraph{Example}
Suppose we want to get the minimum of $f(x_1, x_2) = 2 x_1^2 + 3 x_2^2 - 4 x_1 x_2 + 6 x_1 - 9 x_2 + 2$ using default settings with an initial point $[-1, 0]$ and other default settings. We can use the function like this:
\begin{minted}[breaklines, breakanywhere]{matlab}
[xval, fval] = HookeJeeves(@(x) 2 * x(1)^2 + 3 * x(2)^2 -4 * x(1) * x(2) + 6 * x(1) - 9 * x(2) + 2, [-1, 0])
\end{minted}

And the function returns $\mathbf{x} = [-0.0000, 1.5000], f(\mathbf{x}) = -4.7500$ as a result which is the right answer.

\subsubsection{Rosenbrock Method}
\begin{minted}[breaklines, breakanywhere]{matlab}
function [xval, fval] = Rosenbrock(funct, initial, [epsilon], [dirMatrix], [delta], [alpha], [beta])    
\end{minted}

\paragraph{Introduction}
Find the vector which reaches the minimum value in a function. This function doesn't need derivative of \textcolor{blue}{funct}. And \textcolor{blue}{funct, initial, epsilon, dirMatrix, alpha, beta} is the same as what's mentioned in the Hooke Jeeves Method part. \textcolor{blue}{delta} is a vector which has the same size as \textcolor{blue}{initial}. An orthogonalization is conducted after a new direction is calculated. 

\paragraph{Algorithm}
The algorithm is implemented below, \textcolor{blue}{funct} is denoted as $f$, \textcolor{blue}{initial} is denoted as $\mathbf{x}$, \textcolor{blue}{epsilon} is denoted as $\varepsilon$, \textcolor{blue}{dirMatrix} is denoted as $A$, \textcolor{blue}{delta} is denoted as $\mathbf{\delta}$, \textcolor{blue}{alpha} is denoted as $\alpha$, \textcolor{blue}{beta} is denoted as $\beta$:
\begin{enumerate}
    \item Let a temporary variable $\mathbf{y} = \mathbf{x}$. If $||\mathbf{\delta}|| \geqslant \varepsilon$, go to Step 2, else go to Step 6.
    \item Let $\mathbf{y'} = \mathbf{y}$, In every direction, if $f(\mathbf{y} + \mathbf{\delta}_i A_i) < f(\mathbf{y})$, let $\mathbf{\delta}_i = \alpha \mathbf{\delta}_i$ and $\mathbf{y} = \mathbf{y} + \mathbf{\delta}_i A_i$, else $\mathbf{\delta}_i = \beta \mathbf{\delta}_i$.
    \item If $f(\mathbf{y}) = f(\mathbf{y'})$ and $f(\mathbf{y}) < f(\mathbf{x})$, go to Step 4, else go to Step 1.
    \item Let $\mathbf{x'} = \mathbf{y}, \mathbf{d} = \mathbf{x'} - \mathbf{x}, \mathbf{\lambda}_i = \mathbf{d} \frac{A_i}{||A_i||}$. New direction $A^*$ is calculated using the following formula (nvars means 'number of variables'):
    \begin{equation*}
        \left\{
        \begin{aligned}
            & A^*_i = A_i, \quad \mathbf{\lambda}_i = 0 \\
            & A^*_i = \sum_{j = i}^{\mbox{nvars of $x$}}{\mathbf{\lambda}_j A_j}, \quad \mbox{Otherwise}
        \end{aligned}
        \right.
    \end{equation*}
    \item Orthogonalize $A^*$ using Gram-Schmidt orthogonalization. Update $A$ with that result, and go to Step 1.
    \item Return $\mathbf{x}, f(\mathbf{x})$ as outputs.
\end{enumerate}

\paragraph{\textcolor{red}{Warnings}}
\begin{enumerate}
    \item \textcolor{blue}{alpha} must be greater than $1$. Its default value is $2$.
    \item \textcolor{blue}{beta} must be between $-1$ and $0$, excluding $-1$ and $0$. Its default value is $-0.5$.
    \item \textcolor{blue}{delta} must be the same size as \textcolor{blue}{initial}. Its default value is $[1, 1, \cdots, 1]$.
    \item \textcolor{blue}{epsilon} must be greater than $0$. Its default value is $10^{-6}$.
    \item \textcolor{blue}{dirMatrix} must be a $n \times n$ matrix. Its default value is $A = I$.
\end{enumerate}

\paragraph{Example}
Suppose we want to get the minimum of $f(x_1, x_2) = 2 x_1^2 + 3 x_2^2 - 4 x_1 x_2 + 6 x_1 - 9 x_2 + 2$ using default settings with an initial point $[-1, 0]$ and other default settings. We can use the function like this:
\begin{minted}[breaklines, breakanywhere]{matlab}
[xval, fval] = Rosenbrock(@(x) 2 * x(1)^2 + 3 * x(2)^2 -4 * x(1) * x(2) + 6 * x(1) - 9 * x(2) + 2, [-1, 0])
\end{minted}

And the function returns $\mathbf{x} = [0.0000, 1.5000], f(\mathbf{x}) = -4.7500$ as a result which is the right answer.

\subsubsection{Powell and Powell-Sargent Method}
\begin{minted}[breaklines, breakanywhere]{matlab}
function [xval, fval] = PowellMethod(funct, initial, [epsilon], [step], [dirMatrix])
function [xval, fval] = PowellSargentMethod(funct, initial, [epsilon], [step], [dirMatrix])
\end{minted}

\paragraph{Introduction}
Find the vector which reaches the minimum value in a function. This function doesn't need derivative of \textcolor{blue}{funct}. However, it still needs to do a search to find an optimal step. \textcolor{blue}{initial} is the initial point, \textcolor{blue}{epsilon, step} is the same as what's mentioned in the Search part. \textcolor{blue}{dirMatrix} is the initial direction matrix.

\paragraph{Algorithm}
The algorithm is implemented below, \textcolor{blue}{funct} is denoted as $f$, \textcolor{blue}{initial} is denoted as $\mathbf{x}$, \textcolor{blue}{epsilon} is denoted as $\varepsilon$, \textcolor{blue}{dirMatrix} is denoted as $A$, \textcolor{blue}{step} is denoted as $l$:
\begin{enumerate}
    \item Let $\mathbf{x'} = +\inf$.
    \item If $||\mathbf{x'} - \mathbf{x}|| \geqslant \varepsilon$, go to Step 3, else go to Step 6.
    \item Let $\mathbf{x'} = \mathbf{x}$. Search for the minimum of $f$ in every direction $A_i$ using Golden Mean Method. Update $\mathbf{x}$ at the same time.
    \item Update searching direction matrix: $A_{i + 1} = A_i, i = 1, 2, \cdots, n - 1$ ($A$ is a $n \times n$ matrix).
    \item For the last row of $A$, $A_n = \mathbf{x'} - \mathbf{x}$. Search for the minimum of $f$ in every direction $A_i$ using Golden Mean Method. Update $\mathbf{x}$ at the same time. Go to Step 2.
    \item Return $\mathbf{x}, f(\mathbf{x})$ as outputs.
\end{enumerate}

Powell-Sargent Method is a variation of the Powell Method. 
\begin{enumerate}
    \item In Step 3, find out the index which $f$ has the largest decrement, and store the largest decrement as $D$.
    \item In Step 5, replace the index direction in Step 3 if the following criterion is satisfied:
    \begin{equation*}
        |\lambda| > \sqrt{\frac{f(\mathbf{x}) - f(\mathbf{x'})}{D}}
    \end{equation*}
\end{enumerate}

\paragraph{\textcolor{red}{Warnings}}
\begin{enumerate}
    \item \textcolor{blue}{epsilon} must be greater than $0$. Its default value is $10^{-6}$.
    \item \textcolor{blue}{step} must be greater than $0$. Its default value is $10^{-3}$.
    \item \textcolor{blue}{dirMatrix} must be a $n \times n$ matrix. Its default value is $A = I$.
    \item Some functions are unable to be optimized. If you meet an error says 'Infinite boundary found'. It's likely that the function doesn't have a minimum (which means it minimum is $-\infty$).
    \item Try to make $\varepsilon$ larger if it takes a long time to calculate (or run into a dead loop). If search step $\lambda$ reaches $0$, we will forcibly terminate the iteration.
    \item \textcolor{red}{[Bug Warning]} As we can't process unbounded functions well in this method now, a bound is set to let the program run.  
\end{enumerate}

\paragraph{Example}
Suppose we want to get the minimum of $f(x_1, x_2) = 2 x_1^2 + 3 x_2^2 - 4 x_1 x_2 + 6 x_1 - 9 x_2 + 2$ using default settings with an initial point $[-1, 0]$ and other default settings. We can use the function like this:
\begin{minted}[breaklines, breakanywhere]{matlab}
[xval, fval] = PowellMethod(@(x) 2 * x(1)^2 + 3 * x(2)^2 -4 * x(1) * x(2) + 6 * x(1) - 9 * x(2) + 2, [-1, 0])
[xval, fval] = PowellSargentMethod(@(x) 2 * x(1)^2 + 3 * x(2)^2 -4 * x(1) * x(2) + 6 * x(1) - 9 * x(2) + 2, [-1, 0])
\end{minted}

And the function returns $\mathbf{x} = [0.0000, 1.5000], f(\mathbf{x}) = -4.7500$ as a result which is the right answer.

Powell-Sargent Method is more stable than the original Powell Method.

\section{Constrained Optimization}
\subsection{Barrier Methods}
\textcolor{red}{Warning:} Barrier methods are very instable, it's not recommended to use all of the functions here.

\subsubsection{Outer Barrier Method}
\begin{minted}[breaklines, breakanywhere]{matlab}
function [xval, fval] = OuterBarrier(funct, initial, constraint, equality, [sigma], [epsilon], [step])
\end{minted}
\paragraph{Introduction}
Outer Barrier Method can solve the following problem:
\begin{align*}
    \mbox{minimize } & f(\mathbf{x}) \\
    \mbox{s.t. } & g_i(\mathbf{x}) \geqslant 0, \quad i = 1, 2, \cdots, m \\
                 & h_j(\mathbf{x}) = 0, \quad j = 1, 2, \cdots, l \\
\end{align*}

\textcolor{blue}{constraint} is a cell of inequality constraints, \textcolor{blue}{equality} is a cell of equality constraints, \textcolor{blue}{sigma} is the size to set up a barrier, \textcolor{blue}{funct, initialstep, epsilon} is the same as what's mentioned before. This function is based on Get Outer Barrier function.

\paragraph{Algorithm}
The algorithm is implemented below, \textcolor{blue}{funct} is denoted as $f$, \textcolor{blue}{initial} is denoted as $\mathbf{x}$, \textcolor{blue}{constraint} is denoted as $g_i$, \textcolor{blue}{equality} is denoted as $h_j$, \textcolor{blue}{sigma} is denoted as $\sigma$,  \textcolor{blue}{epsilon} is denoted as $\varepsilon$, \textcolor{blue}{step} is denoted as $l$:
\begin{enumerate}
    \item Constrant barrier function $P(\mathbf{x})$ by $g, h$:
    \begin{align*}
        & P(\mathbf{x}) = \sum_{i = 1}^{m}{\varphi(g_i(\mathbf{x}))} + \sum_{j = 1}^{l}{\psi(h_j(\mathbf{x}))} \\
        & \varphi = (\max\{0, -g_i(\mathbf{x})\})^2 \\
        & \psi = h_j^2(\mathbf{x})
    \end{align*}
    \item If $\sigma P(\mathbf{x}) \geqslant \varepsilon$, go to Step 3, else go to Step 7.
    \item Convert the constrained problem to unconstrained problem:
    \begin{equation*}
        \mbox{minimize } F(\mathbf{x}, \sigma) = f(\mathbf{x}) + \sigma P(\mathbf{x})
    \end{equation*}
    \item Solve the following unconstrained problem using Conjugate Gradient Method, update $\mathbf{x'} = \mathbf{x}$.
    \item Using an adaptive method to update $\sigma, l$. Set a factor $q = 5 + [(\log_{10}{(f(\mathbf{x'}) - f(\mathbf{x}))}) / {3}]$. If $q < 1$, let $q = 1, l = 0.1l$. ($[a]$ means floor function of $a$)
    \item Update $\sigma = q\sigma$, $\mathbf{x} = \mathbf{x'}$, and go to Step 2.
    \item Return $\mathbf{x}, f(\mathbf{x})$ as outputs.
\end{enumerate}

\paragraph{\textcolor{red}{Warnings}}
\begin{enumerate}
    \item \textcolor{blue}{sigma} must be greater than $0$. Its default value is $1000$.
    \item \textcolor{blue}{epsilon} must be greater than $0$. Its default value is $10^{-6}$.
    \item \textcolor{blue}{step} must be greater than $0$. Its default value is $10^{-3}$.
    \item Conjugate Gradient in the unconstrained problem part can be changed with other methods by changing the code:
    \begin{minted}[breaklines, breakanywhere]{matlab}
[xval, ~] = ConjugateGradient(opt_funct, xval, epsilon, step);   
    \end{minted}
\end{enumerate}

\subsubsection{Inner Barrier Method}
\begin{minted}[breaklines, breakanywhere]{matlab}
function [xval, fval] = InnerBarrier(funct, initial, constraint, [type], [beta], [epsilon], [step])
\end{minted}
\paragraph{Introduction}
Inner Barrier Method can solve the following problem:
\begin{align*}
    \mbox{minimize } & f(\mathbf{x}) \\
    \mbox{s.t. } & g_i(\mathbf{x}) \geqslant 0, \quad i = 1, 2, \cdots, m
\end{align*}

\textcolor{blue}{constraint} is a cell of inequality constraints, \textcolor{blue}{beta} is the size to set up a barrier, \textcolor{blue}{funct, initialstep, epsilon} is the same as what's mentioned before. This function is based on Get Inner Barrier function.

\paragraph{Algorithm}
The algorithm is implemented below, \textcolor{blue}{funct} is denoted as $f$, \textcolor{blue}{initial} is denoted as $\mathbf{x}$, \textcolor{blue}{constraint} is denoted as $g_i$, \textcolor{blue}{beta} is denoted as $\beta$,  \textcolor{blue}{epsilon} is denoted as $\varepsilon$, \textcolor{blue}{step} is denoted as $l$:
\begin{enumerate}
    \item Constrant barrier function $B(\mathbf{x})$ by $g$. If type is set to $1$:
    \begin{equation*}
        B(\mathbf{x}) = \sum_{i = 1}^{m}{\frac{1}{g_i(\mathbf{x})}}
    \end{equation*}
    If type is set to $2$:
    \begin{equation*}
        B(\mathbf{x}) = \sum_{i = 1}^{m}{\ln{(g_i(\mathbf{x}))}}
    \end{equation*}
    \item If $\beta B(\mathbf{x}) \geqslant \varepsilon$, go to Step 3, else go to Step 7.
    \item Convert the constrained problem to unconstrained problem:
    \begin{equation*}
        \mbox{minimize } G(\mathbf{x}, \beta) = f(\mathbf{x}) + \beta B(\mathbf{x})
    \end{equation*}
    \item Solve the following unconstrained problem using Conjugate Gradient Method, update $\mathbf{x'} = \mathbf{x}$.
    \item Using an adaptive method to update $\beta, l$. Set a factor $q = 1 / (5 + [(\log_{10}{(f(\mathbf{x'}) - f(\mathbf{x}))}) / {3}])$. If $q > 1$, let $q = 1, l = 0.1l$. ($[a]$ means floor function of $a$)
    \item Update $\beta = q\beta$, $\mathbf{x} = \mathbf{x'}$, and go to Step 2.
    \item Return $\mathbf{x}, f(\mathbf{x})$ as outputs.
\end{enumerate}

\paragraph{\textcolor{red}{Warnings}}
\begin{enumerate}
    \item \textcolor{blue}{beta} must be between $0$ and $1$, excluding $0$. Its default value is $0.5$.
    \item \textcolor{blue}{epsilon} must be greater than $0$. Its default value is $10^{-6}$.
    \item \textcolor{blue}{step} must be greater than $0$. Its default value is $10^{-3}$.
    \item Conjugate Gradient in the unconstrained problem part can be changed with other methods by changing the code:
    \begin{minted}[breaklines, breakanywhere]{matlab}
[xval, ~] = ConjugateGradient(opt_funct, xval, epsilon, step);   
    \end{minted}
\end{enumerate}

\subsection{Problems With Linear Constrants}
\subsubsection{Zoutendijk Method}
\begin{minted}[breaklines, breakanywhere]{matlab}
function [xval, fval] = ZoutendijkMethod(funct, initial, A, b, [Aeq], [beq], [epsilon], [step])
\end{minted}

\paragraph{Introduction}
Zoutendijk Method can solve the following problem:
\begin{align*}
    \mbox{minimize } & f(\mathbf{x}) \\
    \mbox{s.t. } & A\mathbf{x} \geqslant \mathbf{b} \\
                 & A_{\mbox{eq}} \mathbf{x} = \mathbf{b}_{\mbox{eq}}
\end{align*}

In the program, \textcolor{blue}{A, b, Aeq, beq} is the same as what's mentioned in the Linear Programming section, Simplex Solver part, \textcolor{blue}{funct, initial, epsilon, step} is the same as what's mentioned in the Search Functions part.

\paragraph{Algorithm}
The algorithm is implemented below. \textcolor{blue}{A} is denoted as $A$, \textcolor{blue}{b} is denoted as $\mathbf{b}$, \textcolor{blue}{Aeq} is denoted as $E$, \textcolor{blue}{beq} is denoted as $\mathbf{e}$. \textcolor{blue}{funct} is denoted as $f$, and the derivative of $f$ is denoted as $\nabla f$, \textcolor{blue}{initial} is denoted as $\mathbf{x}$, \textcolor{blue}{epsilon} is denoted as $\varepsilon$, \textcolor{blue}{step} is denoted as $l$:
\begin{enumerate}
    \item Seperate $A$. If $A \mathbf{x} = \mathbf{b}$, put it to $A_1, \mathbf{b}_1$, else put it to $A_2, \mathbf{b}_2$.
    \item Solving the following LP using Simplex Solver:
    \begin{align*}
        \mbox{minimize } & \nabla f(\mathbf{x}) \mathbf{d} \\
        \mbox{s.t. } & A_1 \mathbf{d} \geqslant 0 \\
                     & E \mathbf{d} = 0 \\
                     & -1 \leqslant d_j \leqslant 1, \quad j = 1, 2, \cdots, n.
    \end{align*}
    $\mathbf{d}$ is the solution.
    \item If $\nabla f(\mathbf{x}) \mathbf{d} \neq 0$, go to Step 4, else go to Step 7.
    \item Solving the upperbound of $\lambda$ by $A_2 \mathbf{x} + \lambda A_2 \mathbf{d} \geqslant \mathbf{b}_2$. Let it be $\lambda_{\max}$. ($\lambda_{\max}$ can be $+\infty$)
    \item Solving the problem by using the Golden Mean Method:
    \begin{align*}
        \mbox{minimize } & f(\mathbf{x} + \lambda \mathbf{d}) \\
        \mbox{s.t.} & 0 \leqslant \lambda \leqslant \lambda_{\max} 
    \end{align*}
    \item Update $\mathbf{x} = \mathbf{x} + \lambda \mathbf{d}$, and update $A_1, A_2, \mathbf{b}_1, \mathbf{b}_2$ by the new $\mathbf{x}$. Go to Step 3.
    \item Return $\mathbf{x}, f(\mathbf{x})$ as outputs.
\end{enumerate}

\paragraph{\textcolor{red}{Warnings}}
\begin{enumerate}
    \item \textcolor{blue}{epsilon} must be greater than $0$. Its default value is $10^{-6}$.
    \item \textcolor{blue}{step} must be greater than $0$. Its default value is $10^{-3}$.
    \item Please note that you don't need to transpose \textcolor{blue}{initial}. But you still need to make the sizes match.
    \item \textcolor{red}{[Bug Warning]} This program can't cover all of the problems, as the algorithm might not be a closed map.
\end{enumerate}

\paragraph{Example}
Suppose we are goint to solve the following problem:
\begin{align*}
    \mbox{minimize } &  x_1^2 + x_2^2 - 2x_1 - 4x_2 + 6 \\
    \mbox{s.t. } & -2x_1 + x_2 + 1 \geqslant 0 \\
                 & -x_1 - x_2 + 2 \geqslant 0 \\
                 x_1, x_2 \geqslant 0.
\end{align*}

We can use the function like this (with an initial point of $[0, 0]$ and all the default settings):
\begin{minted}[breaklines, breakanywhere]{matlab}
[xval, fval] = ZoutendijkMethod(@(x) x(1)^2 + x(2)^2 - 2 * x(1) - 4 * x(2) + 6, [0, 0], [-2 1; -1 -1; 1 0; 0 1], [-1 -2 0 0]')
\end{minted}

And the function returns xval $=[0.5000, 1.5000]$ and fval $= 1.5000$ as a result, which is the right answer.

\section{Licenses}
\subsection{Project License}
\begin{center}
    The GNU Public License, Version 2
\end{center}

\subsubsection{Preamble}

The licenses for most software are designed to take away your freedom to
share and change it.  By contrast, the \textsc{gnu} General Public License is
intended to guarantee your freedom to share and change free software---to
make sure the software is free for all its users.  This General Public
License applies to most of the Free Software Foundation's software and to
any other program whose authors commit to using it.  (Some other Free
Software Foundation software is covered by the \textsc{gnu} Library
General Public License instead.)  You can apply it to your programs, too.

When we speak of free software, we are referring to freedom, not price.
Our General Public Licenses are designed to make sure that you have the
freedom to distribute copies of free software (and charge for this service
if you wish), that you receive source code or can get it if you want it,
that you can change the software or use pieces of it in new free programs;
and that you know you can do these things.

To protect your rights, we need to make restrictions that forbid anyone to
deny you these rights or to ask you to surrender the rights.  These
restrictions translate to certain responsibilities for you if you
distribute copies of the software, or if you modify it.

For example, if you distribute copies of such a program, whether gratis or
for a fee, you must give the recipients all the rights that you have.  You
must make sure that they, too, receive or can get the source code.  And
you must show them these terms so they know their rights.

We protect your rights with two steps: (1) copyright the software, and (2)
offer you this license which gives you legal permission to copy,
distribute and/or modify the software.

Also, for each author's protection and ours, we want to make certain that
everyone understands that there is no warranty for this free software.  If
the software is modified by someone else and passed on, we want its
recipients to know that what they have is not the original, so that any
problems introduced by others will not reflect on the original authors'
reputations.

Finally, any free program is threatened constantly by software patents.
We wish to avoid the danger that redistributors of a free program will
individually obtain patent licenses, in effect making the program
proprietary.  To prevent this, we have made it clear that any patent must
be licensed for everyone's free use or not licensed at all.

The precise terms and conditions for copying, distribution and
modification follow.

\subsubsection{Terms and Conditions For Copying, Distribution and
  Modification}

\begin{enumerate}

\addtocounter{enumi}{-1}

\item
This License applies to any program or other work which contains a notice
placed by the copyright holder saying it may be distributed under the
terms of this General Public License.  The ``Program'', below, refers to
any such program or work, and a ``work based on the Program'' means either
the Program or any derivative work under copyright law: that is to say, a
work containing the Program or a portion of it, either verbatim or with
modifications and/or translated into another language.  (Hereinafter,
translation is included without limitation in the term ``modification''.)
Each licensee is addressed as ``you''.

Activities other than copying, distribution and modification are not
covered by this License; they are outside its scope.  The act of
running the Program is not restricted, and the output from the Program
is covered only if its contents constitute a work based on the
Program (independent of having been made by running the Program).
Whether that is true depends on what the Program does.

\item You may copy and distribute verbatim copies of the Program's source
  code as you receive it, in any medium, provided that you conspicuously
  and appropriately publish on each copy an appropriate copyright notice
  and disclaimer of warranty; keep intact all the notices that refer to
  this License and to the absence of any warranty; and give any other
  recipients of the Program a copy of this License along with the Program.

You may charge a fee for the physical act of transferring a copy, and you
may at your option offer warranty protection in exchange for a fee.

\item
You may modify your copy or copies of the Program or any portion
of it, thus forming a work based on the Program, and copy and
distribute such modifications or work under the terms of Section 1
above, provided that you also meet all of these conditions:

\begin{enumerate}

\item
You must cause the modified files to carry prominent notices stating that
you changed the files and the date of any change.

\item
You must cause any work that you distribute or publish, that in
whole or in part contains or is derived from the Program or any
part thereof, to be licensed as a whole at no charge to all third
parties under the terms of this License.

\item
If the modified program normally reads commands interactively
when run, you must cause it, when started running for such
interactive use in the most ordinary way, to print or display an
announcement including an appropriate copyright notice and a
notice that there is no warranty (or else, saying that you provide
a warranty) and that users may redistribute the program under
these conditions, and telling the user how to view a copy of this
License.  (Exception: if the Program itself is interactive but
does not normally print such an announcement, your work based on
the Program is not required to print an announcement.)

\end{enumerate}


These requirements apply to the modified work as a whole.  If
identifiable sections of that work are not derived from the Program,
and can be reasonably considered independent and separate works in
themselves, then this License, and its terms, do not apply to those
sections when you distribute them as separate works.  But when you
distribute the same sections as part of a whole which is a work based
on the Program, the distribution of the whole must be on the terms of
this License, whose permissions for other licensees extend to the
entire whole, and thus to each and every part regardless of who wrote it.

Thus, it is not the intent of this section to claim rights or contest
your rights to work written entirely by you; rather, the intent is to
exercise the right to control the distribution of derivative or
collective works based on the Program.

In addition, mere aggregation of another work not based on the Program
with the Program (or with a work based on the Program) on a volume of
a storage or distribution medium does not bring the other work under
the scope of this License.

\item
You may copy and distribute the Program (or a work based on it,
under Section 2) in object code or executable form under the terms of
Sections 1 and 2 above provided that you also do one of the following:

\begin{enumerate}

\item
Accompany it with the complete corresponding machine-readable
source code, which must be distributed under the terms of Sections
1 and 2 above on a medium customarily used for software interchange; or,

\item
Accompany it with a written offer, valid for at least three
years, to give any third party, for a charge no more than your
cost of physically performing source distribution, a complete
machine-readable copy of the corresponding source code, to be
distributed under the terms of Sections 1 and 2 above on a medium
customarily used for software interchange; or,

\item
Accompany it with the information you received as to the offer
to distribute corresponding source code.  (This alternative is
allowed only for noncommercial distribution and only if you
received the program in object code or executable form with such
an offer, in accord with Subsubsection b above.)

\end{enumerate}


The source code for a work means the preferred form of the work for
making modifications to it.  For an executable work, complete source
code means all the source code for all modules it contains, plus any
associated interface definition files, plus the scripts used to
control compilation and installation of the executable.  However, as a
special exception, the source code distributed need not include
anything that is normally distributed (in either source or binary
form) with the major components (compiler, kernel, and so on) of the
operating system on which the executable runs, unless that component
itself accompanies the executable.

If distribution of executable or object code is made by offering
access to copy from a designated place, then offering equivalent
access to copy the source code from the same place counts as
distribution of the source code, even though third parties are not
compelled to copy the source along with the object code.

\item
You may not copy, modify, sublicense, or distribute the Program
except as expressly provided under this License.  Any attempt
otherwise to copy, modify, sublicense or distribute the Program is
void, and will automatically terminate your rights under this License.
However, parties who have received copies, or rights, from you under
this License will not have their licenses terminated so long as such
parties remain in full compliance.

\item
You are not required to accept this License, since you have not
signed it.  However, nothing else grants you permission to modify or
distribute the Program or its derivative works.  These actions are
prohibited by law if you do not accept this License.  Therefore, by
modifying or distributing the Program (or any work based on the
Program), you indicate your acceptance of this License to do so, and
all its terms and conditions for copying, distributing or modifying
the Program or works based on it.

\item
Each time you redistribute the Program (or any work based on the
Program), the recipient automatically receives a license from the
original licensor to copy, distribute or modify the Program subject to
these terms and conditions.  You may not impose any further
restrictions on the recipients' exercise of the rights granted herein.
You are not responsible for enforcing compliance by third parties to
this License.

\item
If, as a consequence of a court judgment or allegation of patent
infringement or for any other reason (not limited to patent issues),
conditions are imposed on you (whether by court order, agreement or
otherwise) that contradict the conditions of this License, they do not
excuse you from the conditions of this License.  If you cannot
distribute so as to satisfy simultaneously your obligations under this
License and any other pertinent obligations, then as a consequence you
may not distribute the Program at all.  For example, if a patent
license would not permit royalty-free redistribution of the Program by
all those who receive copies directly or indirectly through you, then
the only way you could satisfy both it and this License would be to
refrain entirely from distribution of the Program.

If any portion of this section is held invalid or unenforceable under
any particular circumstance, the balance of the section is intended to
apply and the section as a whole is intended to apply in other
circumstances.

It is not the purpose of this section to induce you to infringe any
patents or other property right claims or to contest validity of any
such claims; this section has the sole purpose of protecting the
integrity of the free software distribution system, which is
implemented by public license practices.  Many people have made
generous contributions to the wide range of software distributed
through that system in reliance on consistent application of that
system; it is up to the author/donor to decide if he or she is willing
to distribute software through any other system and a licensee cannot
impose that choice.

This section is intended to make thoroughly clear what is believed to
be a consequence of the rest of this License.

\item
If the distribution and/or use of the Program is restricted in
certain countries either by patents or by copyrighted interfaces, the
original copyright holder who places the Program under this License
may add an explicit geographical distribution limitation excluding
those countries, so that distribution is permitted only in or among
countries not thus excluded.  In such case, this License incorporates
the limitation as if written in the body of this License.

\item
The Free Software Foundation may publish revised and/or new versions
of the General Public License from time to time.  Such new versions will
be similar in spirit to the present version, but may differ in detail to
address new problems or concerns.

Each version is given a distinguishing version number.  If the Program
specifies a version number of this License which applies to it and ``any
later version'', you have the option of following the terms and conditions
either of that version or of any later version published by the Free
Software Foundation.  If the Program does not specify a version number of
this License, you may choose any version ever published by the Free Software
Foundation.

\item
If you wish to incorporate parts of the Program into other free
programs whose distribution conditions are different, write to the author
to ask for permission.  For software which is copyrighted by the Free
Software Foundation, write to the Free Software Foundation; we sometimes
make exceptions for this.  Our decision will be guided by the two goals
of preserving the free status of all derivatives of our free software and
of promoting the sharing and reuse of software generally.

\end{enumerate}

\subsubsection{No Warranty}

\begin{enumerate}
\item
Because the program is licensed free of charge, there is no warranty
for the program, to the extent permitted by applicable law.  Except when
otherwise stated in writing the copyright holders and/or other parties
provide the program ``as is'' without warranty of any kind, either expressed
or implied, including, but not limited to, the implied warranties of
merchantability and fitness for a particular purpose.  The entire risk as
to the quality and performance of the program is with you.  Should the
program prove defective, you assume the cost of all necessary servicing,
repair or correction.

\item
In no event unless required by applicable law or agreed to in writing
will any copyright holder, or any other party who may modify and/or
redistribute the program as permitted above, be liable to you for damages,
including any general, special, incidental or consequential damages arising
out of the use or inability to use the program (including but not limited
to loss of data or data being rendered inaccurate or losses sustained by
you or third parties or a failure of the program to operate with any other
programs), even if such holder or other party has been advised of the
possibility of such damages.
\end{enumerate}

\subsection{Document License}
\begin{center}
    GNU Free Documentation License, Version 1.2, November 2002
\end{center}
\subsubsection{Preamble}

The purpose of this License is to make a manual, textbook, or other
functional and useful document ``free'' in the sense of freedom: to
assure everyone the effective freedom to copy and redistribute it,
with or without modifying it, either commercially or noncommercially.
Secondarily, this License preserves for the author and publisher a way
to get credit for their work, while not being considered responsible
for modifications made by others.

This License is a kind of ``copyleft'', which means that derivative
works of the document must themselves be free in the same sense.  It
complements the GNU General Public License, which is a copyleft
license designed for free software.

We have designed this License in order to use it for manuals for free
software, because free software needs free documentation: a free
program should come with manuals providing the same freedoms that the
software does.  But this License is not limited to software manuals;
it can be used for any textual work, regardless of subject matter or
whether it is published as a printed book.  We recommend this License
principally for works whose purpose is instruction or reference.

\subsubsection{Applicability and definitions}

This License applies to any manual or other work, in any medium, that
contains a notice placed by the copyright holder saying it can be
distributed under the terms of this License.  Such a notice grants a
world-wide, royalty-free license, unlimited in duration, to use that
work under the conditions stated herein.  The \textbf{``Document''}, below,
refers to any such manual or work.  Any member of the public is a
licensee, and is addressed as \textbf{``you''}.  You accept the license if you
copy, modify or distribute the work in a way requiring permission
under copyright law.

A \textbf{``Modified Version''} of the Document means any work containing the
Document or a portion of it, either copied verbatim, or with
modifications and/or translated into another language.

A \textbf{``Secondary Section''} is a named appendix or a front-matter section of
the Document that deals exclusively with the relationship of the
publishers or authors of the Document to the Document's overall subject
(or to related matters) and contains nothing that could fall directly
within that overall subject.  (Thus, if the Document is in part a
textbook of mathematics, a Secondary Section may not explain any
mathematics.)  The relationship could be a matter of historical
connection with the subject or with related matters, or of legal,
commercial, philosophical, ethical or political position regarding
them.

The \textbf{``Invariant Sections''} are certain Secondary Sections whose titles
are designated, as being those of Invariant Sections, in the notice
that says that the Document is released under this License.  If a
section does not fit the above definition of Secondary then it is not
allowed to be designated as Invariant.  The Document may contain zero
Invariant Sections.  If the Document does not identify any Invariant
Sections then there are none.

The \textbf{``Cover Texts''} are certain short passages of text that are listed,
as Front-Cover Texts or Back-Cover Texts, in the notice that says that
the Document is released under this License.  A Front-Cover Text may
be at most 5 words, and a Back-Cover Text may be at most 25 words.

A \textbf{``Transparent''} copy of the Document means a machine-readable copy,
represented in a format whose specification is available to the
general public, that is suitable for revising the document
straightforwardly with generic text editors or (for images composed of
pixels) generic paint programs or (for drawings) some widely available
drawing editor, and that is suitable for input to text formatters or
for automatic translation to a variety of formats suitable for input
to text formatters.  A copy made in an otherwise Transparent file
format whose markup, or absence of markup, has been arranged to thwart
or discourage subsequent modification by readers is not Transparent.
An image format is not Transparent if used for any substantial amount
of text.  A copy that is not ``Transparent'' is called \textbf{``Opaque''}.

Examples of suitable formats for Transparent copies include plain
ASCII without markup, Texinfo input format, LaTeX input format, SGML        
or XML using a publicly available DTD, and standard-conforming simple
HTML, PostScript or PDF designed for human modification.  Examples of
transparent image formats include PNG, XCF and JPG.  Opaque formats
include proprietary formats that can be read and edited only by
proprietary word processors, SGML or XML for which the DTD and/or
processing tools are not generally available, and the
machine-generated HTML, PostScript or PDF produced by some word
processors for output purposes only.

The \textbf{``Title Page''} means, for a printed book, the title page itself,
plus such following pages as are needed to hold, legibly, the material
this License requires to appear in the title page.  For works in
formats which do not have any title page as such, ``Title Page'' means
the text near the most prominent appearance of the work's title,
preceding the beginning of the body of the text.

A section \textbf{``Entitled XYZ''} means a named subunit of the Document whose
title either is precisely XYZ or contains XYZ in parentheses following
text that translates XYZ in another language.  (Here XYZ stands for a
specific section name mentioned below, such as \textbf{``Acknowledgements''},
\textbf{``Dedications''}, \textbf{``Endorsements''}, or \textbf{``History''}.)
To \textbf{``Preserve the Title''}
of such a section when you modify the Document means that it remains a
section ``Entitled XYZ'' according to this definition.

The Document may include Warranty Disclaimers next to the notice which
states that this License applies to the Document.  These Warranty
Disclaimers are considered to be included by reference in this
License, but only as regards disclaiming warranties: any other
implication that these Warranty Disclaimers may have is void and has
no effect on the meaning of this License.

\subsubsection{Verbatim Copying}

You may copy and distribute the Document in any medium, either
commercially or noncommercially, provided that this License, the
copyright notices, and the license notice saying this License applies
to the Document are reproduced in all copies, and that you add no other
conditions whatsoever to those of this License.  You may not use
technical measures to obstruct or control the reading or further
copying of the copies you make or distribute.  However, you may accept
compensation in exchange for copies.  If you distribute a large enough
number of copies you must also follow the conditions in section 3.

You may also lend copies, under the same conditions stated above, and
you may publicly display copies.

\subsubsection{Copying in Quantity}

If you publish printed copies (or copies in media that commonly have
printed covers) of the Document, numbering more than 100, and the
Document's license notice requires Cover Texts, you must enclose the
copies in covers that carry, clearly and legibly, all these Cover
Texts: Front-Cover Texts on the front cover, and Back-Cover Texts on
the back cover.  Both covers must also clearly and legibly identify
you as the publisher of these copies.  The front cover must present
the full title with all words of the title equally prominent and
visible.  You may add other material on the covers in addition.
Copying with changes limited to the covers, as long as they preserve
the title of the Document and satisfy these conditions, can be treated
as verbatim copying in other respects.

If the required texts for either cover are too voluminous to fit
legibly, you should put the first ones listed (as many as fit
reasonably) on the actual cover, and continue the rest onto adjacent
pages.

If you publish or distribute Opaque copies of the Document numbering
more than 100, you must either include a machine-readable Transparent
copy along with each Opaque copy, or state in or with each Opaque copy
a computer-network location from which the general network-using
public has access to download using public-standard network protocols
a complete Transparent copy of the Document, free of added material.
If you use the latter option, you must take reasonably prudent steps,
when you begin distribution of Opaque copies in quantity, to ensure
that this Transparent copy will remain thus accessible at the stated
location until at least one year after the last time you distribute an
Opaque copy (directly or through your agents or retailers) of that
edition to the public.

It is requested, but not required, that you contact the authors of the
Document well before redistributing any large number of copies, to give
them a chance to provide you with an updated version of the Document.

\subsubsection{Modifications}

You may copy and distribute a Modified Version of the Document under
the conditions of sections 2 and 3 above, provided that you release
the Modified Version under precisely this License, with the Modified
Version filling the role of the Document, thus licensing distribution
and modification of the Modified Version to whoever possesses a copy
of it.  In addition, you must do these things in the Modified Version:

\begin{itemize}
\item[A.]
   Use in the Title Page (and on the covers, if any) a title distinct
   from that of the Document, and from those of previous versions
   (which should, if there were any, be listed in the History section
   of the Document).  You may use the same title as a previous version
   if the original publisher of that version gives permission.

\item[B.]
   List on the Title Page, as authors, one or more persons or entities
   responsible for authorship of the modifications in the Modified
   Version, together with at least five of the principal authors of the
   Document (all of its principal authors, if it has fewer than five),
   unless they release you from this requirement.

\item[C.]
   State on the Title page the name of the publisher of the
   Modified Version, as the publisher.

\item[D.]
   Preserve all the copyright notices of the Document.

\item[E.]
   Add an appropriate copyright notice for your modifications
   adjacent to the other copyright notices.

\item[F.]
   Include, immediately after the copyright notices, a license notice
   giving the public permission to use the Modified Version under the
   terms of this License, in the form shown in the Addendum below.

\item[G.]
   Preserve in that license notice the full lists of Invariant Sections
   and required Cover Texts given in the Document's license notice.

\item[H.]
   Include an unaltered copy of this License.

\item[I.]
   Preserve the section Entitled ``History'', Preserve its Title, and add
   to it an item stating at least the title, year, new authors, and
   publisher of the Modified Version as given on the Title Page.  If
   there is no section Entitled ``History'' in the Document, create one
   stating the title, year, authors, and publisher of the Document as
   given on its Title Page, then add an item describing the Modified
   Version as stated in the previous sentence.

\item[J.]
   Preserve the network location, if any, given in the Document for
   public access to a Transparent copy of the Document, and likewise
   the network locations given in the Document for previous versions
   it was based on.  These may be placed in the ``History'' section.
   You may omit a network location for a work that was published at
   least four years before the Document itself, or if the original
   publisher of the version it refers to gives permission.

\item[K.]
   For any section Entitled ``Acknowledgements'' or ``Dedications'',
   Preserve the Title of the section, and preserve in the section all
   the substance and tone of each of the contributor acknowledgements
   and/or dedications given therein.

\item[L.]
   Preserve all the Invariant Sections of the Document,
   unaltered in their text and in their titles.  Section numbers
   or the equivalent are not considered part of the section titles.

\item[M.]
   Delete any section Entitled ``Endorsements''.  Such a section
   may not be included in the Modified Version.

\item[N.]
   Do not retitle any existing section to be Entitled ``Endorsements''
   or to conflict in title with any Invariant Section.

\item[O.]
   Preserve any Warranty Disclaimers.
\end{itemize}

If the Modified Version includes new front-matter sections or
appendices that qualify as Secondary Sections and contain no material
copied from the Document, you may at your option designate some or all
of these sections as invariant.  To do this, add their titles to the
list of Invariant Sections in the Modified Version's license notice.
These titles must be distinct from any other section titles.

You may add a section Entitled ``Endorsements'', provided it contains
nothing but endorsements of your Modified Version by various
parties--for example, statements of peer review or that the text has
been approved by an organization as the authoritative definition of a
standard.

You may add a passage of up to five words as a Front-Cover Text, and a
passage of up to 25 words as a Back-Cover Text, to the end of the list
of Cover Texts in the Modified Version.  Only one passage of
Front-Cover Text and one of Back-Cover Text may be added by (or
through arrangements made by) any one entity.  If the Document already
includes a cover text for the same cover, previously added by you or
by arrangement made by the same entity you are acting on behalf of,
you may not add another; but you may replace the old one, on explicit
permission from the previous publisher that added the old one.

The author(s) and publisher(s) of the Document do not by this License
give permission to use their names for publicity for or to assert or
imply endorsement of any Modified Version.

\subsubsection{Combining Documents}

You may combine the Document with other documents released under this
License, under the terms defined in section 4 above for modified
versions, provided that you include in the combination all of the
Invariant Sections of all of the original documents, unmodified, and
list them all as Invariant Sections of your combined work in its
license notice, and that you preserve all their Warranty Disclaimers.

The combined work need only contain one copy of this License, and
multiple identical Invariant Sections may be replaced with a single
copy.  If there are multiple Invariant Sections with the same name but
different contents, make the title of each such section unique by
adding at the end of it, in parentheses, the name of the original
author or publisher of that section if known, or else a unique number.
Make the same adjustment to the section titles in the list of
Invariant Sections in the license notice of the combined work.

In the combination, you must combine any sections Entitled ``History''
in the various original documents, forming one section Entitled
``History''; likewise combine any sections Entitled ``Acknowledgements'',
and any sections Entitled ``Dedications''.  You must delete all sections
Entitled ``Endorsements''.


\subsubsection{Collection of Documents}

You may make a collection consisting of the Document and other documents
released under this License, and replace the individual copies of this
License in the various documents with a single copy that is included in
the collection, provided that you follow the rules of this License for
verbatim copying of each of the documents in all other respects.

You may extract a single document from such a collection, and distribute
it individually under this License, provided you insert a copy of this
License into the extracted document, and follow this License in all
other respects regarding verbatim copying of that document.


\subsubsection{Aggregating with independent Works}

A compilation of the Document or its derivatives with other separate
and independent documents or works, in or on a volume of a storage or
distribution medium, is called an ``aggregate'' if the copyright
resulting from the compilation is not used to limit the legal rights
of the compilation's users beyond what the individual works permit.
When the Document is included in an aggregate, this License does not
apply to the other works in the aggregate which are not themselves
derivative works of the Document.

If the Cover Text requirement of section 3 is applicable to these
copies of the Document, then if the Document is less than one half of
the entire aggregate, the Document's Cover Texts may be placed on
covers that bracket the Document within the aggregate, or the
electronic equivalent of covers if the Document is in electronic form.
Otherwise they must appear on printed covers that bracket the whole
aggregate.



\subsubsection{Translation}

Translation is considered a kind of modification, so you may
distribute translations of the Document under the terms of section 4.
Replacing Invariant Sections with translations requires special
permission from their copyright holders, but you may include
translations of some or all Invariant Sections in addition to the
original versions of these Invariant Sections.  You may include a
translation of this License, and all the license notices in the
Document, and any Warranty Disclaimers, provided that you also include
the original English version of this License and the original versions
of those notices and disclaimers.  In case of a disagreement between
the translation and the original version of this License or a notice
or disclaimer, the original version will prevail.

If a section in the Document is Entitled ``Acknowledgements'',
``Dedications'', or ``History'', the requirement (section 4) to Preserve
its Title (section 1) will typically require changing the actual
title.


\subsubsection{Termination}

You may not copy, modify, sublicense, or distribute the Document except
as expressly provided for under this License.  Any other attempt to
copy, modify, sublicense or distribute the Document is void, and will
automatically terminate your rights under this License.  However,
parties who have received copies, or rights, from you under this
License will not have their licenses terminated so long as such
parties remain in full compliance.


\subsubsection{Future Revisions of this License}

The Free Software Foundation may publish new, revised versions
of the GNU Free Documentation License from time to time.  Such new
versions will be similar in spirit to the present version, but may
differ in detail to address new problems or concerns.  See
http://www.gnu.org/copyleft/.

Each version of the License is given a distinguishing version number.
If the Document specifies that a particular numbered version of this
License ``or any later version'' applies to it, you have the option of
following the terms and conditions either of that specified version or
of any later version that has been published (not as a draft) by the
Free Software Foundation.  If the Document does not specify a version
number of this License, you may choose any version ever published (not
as a draft) by the Free Software Foundation.


\subsubsection{Addendum: How to use this License for your documents}

To use this License in a document you have written, include a copy of
the License in the document and put the following copyright and
license notices just after the title page:

\bigskip
\begin{quote}
    Copyright \copyright \textsc{year your name}.
    Permission is granted to copy, distribute and/or modify this document
    under the terms of the GNU Free Documentation License, Version 1.2
    or any later version published by the Free Software Foundation;
    with no Invariant Sections, no Front-Cover Texts, and no Back-Cover Texts.
    A copy of the license is included in the section entitled ``GNU
    Free Documentation License''.
\end{quote}
\bigskip

If you have Invariant Sections, Front-Cover Texts and Back-Cover Texts,
replace the ``with \dots\ Texts.'' line with this:

\bigskip
\begin{quote}
    with the Invariant Sections being \textsc{list their titles}, with the
    Front-Cover Texts being \textsc{list}, and with the Back-Cover
    Texts being \textsc{list}.
\end{quote}
\bigskip

If you have Invariant Sections without Cover Texts, or some other
combination of the three, merge those two alternatives to suit the
situation.

If your document contains nontrivial examples of program code, we
recommend releasing these examples in parallel under your choice of
free software license, such as the GNU General Public License,
to permit their use in free software.

\end{document}
